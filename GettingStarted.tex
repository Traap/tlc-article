\documentclass[12pt]{tlc-article}

\def\tlcProduct{tlc-article}

\def\tlcA{\tlcDarkblue{\tlcProduct}}
\def\tlcAL{\textit{docstyle/additional-layout.tex}}
\def\tlcHF{\textit{docstyle/header-footer.tex}}

\def\kpse{\$(kpsewhich -var-value TEXMFLOCAL)}
\def\texDist{\kpse}
\def\tlcDist{/tex/latex/\tlcProduct}
\def\tlcGlobalDist{\texDist\tlcDist}

\def\tlcHome{\$HOME}
\def\tlcMyDoc{\tlcHome/mydoc}

\def\gitHub{GitHub.com}
\def\gitHubUrl{http://\gitHub}

% ------------------------------------------------------------------------------
\def\tlcRepo{git@\gitHub:Traap/\tlcProduct.git}

\begin{document}

% ------------------------------------------------------------------------------

\tlcTitlePageAndTableOfContents
  {Getting Started}
  {Gary Allan Howard}
  {Getting Started guide covers how to install \tlcA\ both globally and
    locally, describes the general use case, how to customize your \tlcA\
    environment, describes the commands \tlcA\ implements, and reveals the
    packages \tlcA\ depends upon.}

% ------------------------------------------------------------------------------

\section{Installation}
This section describes how to install \tlcA\ either globally to make it available
to your \LaTeX\ environment or locally to the document you are authoring.  This
section identifies the prerequisites you must meet in order to clone a repository
from GitHub.com and install software on your computer.

\subsection{Prerequisites}
The following prerequisites are needed.
\begin{description}[style=nextline]
  \item[Administrative privilege] You will need administrative privileges to
    install \tlcA\ globally because `sudo' is used.

  \item[SSH key] You will need your private key to access \gitHub.  Please refer
    to \url{http://help.github.com/articles/generating-an-ssh-key} for
    instructions on `Generating an SSH key'.
    
  \item[Enable your SSH key] The following instructions enable your SSH key so
    you to not have to enter the passphrase for each git command.

    \begin{enumerate}
      \item eval \$(ssh-agent -s)
      \item ssh-add ~/.ssh/your-private-key
      \item enter your passphrase
    \end{enumerate}

\end{description}

\subsection{Local installation}
A local installation is done by installing \tlcA\ into
/the/path/to/your/document.  Assuming your document is located at \tlcMyDoc\ the
following shell commands will make \tlcA\ available to your document.

\begin{enumerate}
  \item cd \tlcHome
  \item git clone \tlcRepo\
  \item cd \tlcProduct\ 
  \item mkdir \tlcMyDoc
  \item cp -v \tlcProduct.cls \tlcMyDoc/.
\end{enumerate}

\clearpage
\subsection{Global installation}
A global installation is done by installing \tlcA\ into your /path/to/your/texmf
directory.  Assuming a TexLive installation exists at \texDist\ the following
shell commands will make \tlcA\ available to your \LaTeX\ environment.

\begin{enumerate}
  \item cd \tlcHome
  \item git clone \tlcRepo\
  \item cd \tlcProduct\
  \item sudo mkdir -p \tlcGlobalDist
  \item sudo mv -v \tlcProduct.cls \tlcGlobalDist/.
  \item sudo mktexlsr \texDist\
\end{enumerate}

\bigskip
\textbf{Note:} You may remove your local installation by removing \tlcProduct.

% ------------------------------------------------------------------------------

\section{General Use Case}
\subsection{documentclass}

% ------------------------------------------------------------------------------

\section{Customization}
This section describes how \tlcA\ can be customized by using the file-hooks
\tlcA\ check for.  \tlcA\ default implementation will be used when the
file-hooks are now found.

% ------------------------------------------------------------------------------

\subsection{\tlcAL}
\tlcA\ will use whatever \LaTeX\ definitions are found in \tlcAL\ when it
exists.  The file-check is shown
below:

\begin{lstlisting}[basicstyle=\tiny]
  \IfFileExists{docstyle/additional-layout.tex}
    {% This use case demonstrates tlc-article being extended.  All definitions are
% process during preamble phase.  In other words, before your \begin{document}
% statement.

% ------------------------------------------------------------------------------
% \makeatletter is used so we can reference commands and definitions defined by
% tlc-article, which are all prefaced with tlc@.
\makeatletter

% ------------------------------------------------------------------------------
% tlc-article.tex (Getting Starting) definitions.

\def\tlcProduct{tlc-article}%

\def\tlcA{\tlcDarkblue{\tlcProduct}}%

\def\tlcAL{\tlcDarkblue{\tlc@additionalLayout}}%
\def\tlcBL{\tlcDarkblue{tlcBeginLandscape}}%
\def\tlcDB{\tlcDarkblue{tlcDarkblue}}%
\def\tlcEL{\tlcDarkblue{tlcEndLandScape}}%
\def\tlcHF{\tlcDarkblue{\tlc@headerFooter}}%
\def\tlcLG{\tlcDarkblue{\tlc@logoFile}}%
\def\tlcNCT{newcolumn type: \tlcDarkblue{L, C} \& \tlcDarkblue{R}}%
\def\tlcTOC{\tlcDarkblue{tlcTitlePageAndTableOfContents}}%
\def\tlcVE{\tlcDarkblue{\tlc@versionFile}}%

\def\tlcVC{\tlcDarkblue{tlc@version}}%
\def\tlcDC{\tlcDarkblue{tlc@date}}%
\def\tlcSC{\tlcDarkblue{tlc@status}}%
\def\tlcIC{\tlcDarkblue{tlc@instatution}}%
\def\tlcPC{\tlcDarkblue{tlc@permission}}%

\def\kpse{\$(kpsewhich -var-value TEXMFLOCAL)}%
\def\texDist{\kpse}%
\def\tlcDist{/tex/latex/\tlcProduct}%
\def\tlcGlobalDist{\texDist\tlcDist}%

\def\tlcHome{\$HOME}%
\def\tlcMyDoc{\tlcHome/mydoc}%

\def\gitHub{GitHub.com}%
\def\gitHubUrl{http://\gitHub}%

\def\tlcRepo{git@\gitHub:Traap/\tlcProduct.git}%

\def\tlcPkgFile{data/required-packages.csv}%
\def\tlcNote{\tlcDarkblue{Note}}%

% ------------------------------------------------------------------------------%
% Define the column names used by csvreader when reading \packageFile.
\csvnames{tlcPkgNames}{
   1=\name
  ,2=\description
}

% Define the table style used to report the required package names and
% descriptions.
\csvstyle{tlcPkgStyle}{
  longtable=|L{3cm}|L{12cm}|
  ,table head=\hline Name & Description\\\hline\hline\endhead
  ,late after line=\\\hline
  ,tlcPkgNames
}

% ------------------------------------------------------------------------------%
}
    {}
\end{lstlisting}

% ------------------------------------------------------------------------------

\subsection{\tlcHF}
\tlcA\ has a builtin header and footer strategy that is base on
\textit{fancyhdr}, \textit{titling}, and \textit{lastpage} \LaTeX\ packages.
The default implementation is show below:

\begin{lstlisting}[basicstyle=\tiny]
  \IfFileExists{docstyle/header-footer.tex}
    {\input{docstyle/header-footer.tex}}
    {
    % We provide a default implementation for lhead, chead, rhead, 
    % lfoot, cfoot, and rfoot.
    \newcommand*{\tlc@Lhead}{\tiny \tlcDarkblue{fancyndr.lhead}}
    \newcommand*{\tlc@Chead}{\large\tlcDarkblue{fancyndr.chead}}
    \newcommand*{\tlc@Rhead}{\tiny \tlcDarkblue{fancyndr.rhead}}
    \newcommand*{\tlc@Lfoot}{\tiny \tlcDarkblue{fancyndr.lfoot}}
    \newcommand*{\tlc@Cfoot}{\tiny \tlcDarkblue{fancyndr.cfoot}}
    \newcommand*{\tlc@Rfoot}{\tiny \tlcDarkblue{Page~\thepage~of~\pageref{LastPage}}}

    % We want our table of contents to use dots as a leader.
    \RequirePackage{tocloft}
    \renewcommand\cftsecleader{\cftdotfill{\cftdotsep}}
    % Packages needed to create nice looking headers and footers.
    \RequirePackage{fancyhdr}   % Page layout in \LaTeX
    \RequirePackage{titling}    % Control over \maketitle & \thanks
    \RequirePackage{lastpage}   % Page n of m
    \pagestyle{fancy}           % fancy page style
    % Header applied to all pages.
    \lhead{\tlc@Lhead}
    \chead{\tlc@Chead}
    \rhead{\tlc@Rhead}
    \renewcommand{\headrulewidth}{0.1pt}
    % Eliminate head height too small warning, which is occurring because
    % we are using multiple lines in our header.
    \setlength\headheight{18pt}
    % Footer applied to all pages.
    \lfoot{\tlc@Lfoot}
    \cfoot{\tlc@Cfoot}
    \rfoot{\tlc@Rfoot}
    \renewcommand{\footrulewidth}{0.1pt}
    % We want our header and footer to remain consistent with a table of
    % contents that span multiple pages.
    \AtBeginDocument{\addtocontents{toc}{\protect\thispagestyle{fancy}}}
    }
\end{lstlisting}

The default implementation can be overridden by defining \tlcHF. \textbf{Note:}
When \tlcHF\ exists and is empty, your document will be typeset with the
defaults from document-class article.

% ------------------------------------------------------------------------------

\section{Commands}
\subsection{tlcBeginLandscape}
\subsection{tlcDarkblue}
\subsection{tlcEndLandscape}
\subsection{tlcTitlePageAndTableOfContents}

% ------------------------------------------------------------------------------
\clearpage
\section{Required Packages}
This section documents the dependencies of the required package tlc-article has.
Package names are listed in alphabetical order. A complete description of each
package is found at \url{http://www.ctan.org/}. At this writing, you can type in the
package name and press the search button to learn more about each package.

% Define the package file name.
\def\pkgFile{required-packages.csv}

% Define the column names used by csvreader when reading \packageFile.
\csvnames{pkgNames}{
   1=\name
  ,2=\description
}

% Define the table style used to report the required package names and
% descriptions.
\csvstyle{pkgStyle}{
  longtable=|L{3cm}|L{12cm}|
  ,table head=\hline Name & Description\\\hline\hline\endhead
  ,late after line=\\\hline
  ,pkgNames
}

% Render pckFile using pckStyle as a longtable.
\csvreader[pkgStyle, separator=pipe]{\pkgFile}{}{\name & \description}
% ------------------------------------------------------------------------------

\end{document}
